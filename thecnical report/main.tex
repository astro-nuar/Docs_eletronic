\documentclass[12pt]{article}
% ----------------------------------------------------------------------
% Define external packages, language, margins, fonts, new commands 
% and colors
% ----------------------------------------------------------------------
\usepackage[utf8]{inputenc} % Codification
\usepackage[english]{babel} % Writing idiom

\usepackage[export]{adjustbox} % Align images
\usepackage{amsmath} % Extra commands for math mode
\usepackage{amssymb} % Mathematical symbols
\usepackage{anysize} % Personalize margins
    \marginsize{2cm}{2cm}{2cm}{2cm} % {left}{right}{above}{below}
\usepackage{appendix} % Appendices
\usepackage{cancel} % Expression cancellation
\usepackage{caption} % Captions
    \captionsetup{labelfont={bf}}
\usepackage{color} % Text coloring (xcolor is loaded below too)
\usepackage{xcolor} % preferred color package

% --------- define colors early (used by other packages) ----------
% Exact colors from NB
\definecolor{novablue}{RGB}{0,101,189}
\definecolor{dkgreen}{RGB}{52,129,65}
\definecolor{cellborder}{HTML}{CFCFCF}
\definecolor{cellbackground}{HTML}{F7F7F7}
\definecolor{incolor}{HTML}{303F9F}
\definecolor{outcolor}{HTML}{D84315}

% Now hyperref can safely use novablue
\usepackage[colorlinks = true, plainpages = true,
            linkcolor = novablue, urlcolor = novablue,
            citecolor = novablue, anchorcolor = novablue]{hyperref}

\usepackage{fancyhdr} % Head note and footnote
    \pagestyle{fancy}
    \fancyhf{}
    \fancyhead[L]{
        \includegraphics[scale = 0.3]{Images/NovaFctHor.png}
    } % Left of Head note
    \fancyhead[R]{\footnotesize Cadeira} % Right of Head note
    \fancyfoot[L]{\footnotesize MEEC} % Left of Footnote
    \fancyfoot[R]{\thepage} % Right of Footnote
    \renewcommand{\footrulewidth}{0.4pt} % Footnote rule

\usepackage{float} % Utilization of [H] in figures
\usepackage{graphicx} % Figures in LaTeX
\usepackage{indentfirst} % First paragraph
\usepackage[super]{nth} % Superscripts
\usepackage{siunitx} % SI units
\usepackage{subcaption} % Subfigures
\usepackage{titlesec} % Font
    \titleformat{\section}{\Large\bfseries}{\thesection}{1em}{}
    \titleformat{\subsection}{\large\bfseries}{\thesubsection}{1em}{}
    \titleformat{\subsubsection}{\normalsize\bfseries}{\thesubsubsection}{1em}{}

% code listings and related packages
\usepackage[breakable]{tcolorbox}
\usepackage[numbers]{natbib}
\usepackage[acronyms]{glossaries}

% Code highlighting: set after color definitions
\usepackage{listings}
\lstset{
    backgroundcolor=\color{cellbackground}, % background color for the code block
    basicstyle=\ttfamily,                   % font style
    breakatwhitespace=false,                % automatic breaks only at whitespace
    breaklines=true,                        % automatic line breaking
    captionpos=b,                           % caption position
    commentstyle=\color{gray},              % comment style
    escapeinside={(*@}{@*)},                % safer delimiters for LaTeX inside code
    keywordstyle=\color{blue},              % keyword style
    stringstyle=\color{dkgreen},            % string style
    frame=single,                           % adds a frame around the code
    rulecolor=\color{cellborder},           % border color
}

%---------TIKZ CONFIG (use tikzset) ---------%
\usepackage{tikz}
\usetikzlibrary{shapes.geometric, arrows}
\tikzset{
  block/.style = {draw, rectangle, minimum height=2.5em, minimum width=3.5em},
  arrow/.style  = {thick,->,>=stealth},
  startstop/.style = {rectangle, rounded corners, minimum width=3cm, minimum height=1cm, text centered, draw=black, fill=blue!30},
  process/.style   = {rectangle, minimum width=3cm, minimum height=1cm, text centered, draw=black, fill=cyan!30},
  decision/.style  = {diamond, minimum width=3cm, minimum height=1cm, text centered, draw=black, fill=lightblue}
}
% define lightblue once (used in tikzset)
\definecolor{lightblue}{RGB}{173,216,230}

% Tables
\usepackage{tabularx} 
\usepackage{booktabs}
\usepackage{makecell}

% New and re-newcommands
% Fix: do NOT call \sen recursively
\newcommand{\sen}{\operatorname{sen}} % Sine function definition (Portuguese 'sen')

\newcommand{\HRule}{\rule{\linewidth}{0.5mm}} % Specific rule definition
\renewcommand{\appendixpagename}{\LARGE Appendices}

\makeglossaries

%%%%%%%%%%%%%%%%%%%%%%%%%%%%%%%%%%%%%%%%%%%%%%%%%%%%%%%%%%%%%%%%%%%%%%%%
%                                 Document                             %
%%%%%%%%%%%%%%%%%%%%%%%%%%%%%%%%%%%%%%%%%%%%%%%%%%%%%%%%%%%%%%%%%%%%%%%%

\newacronym{adcs}{ADCS}{Attitude Determination Control System}
\newacronym{isa}{ISA}{International Standard Atmosphere}
\newacronym{mekf}{MEKF}{Multiplicative Extended Kalman Filter}
\newacronym{mcu}{MCU}{Microcontroller Unit}
\newacronym{pid}{PID}{Product Integral Derivative}
\newacronym{euroc}{EuRoC}{European Rocketry Challenge}

\begin{document}

% Add bibliography data and style
\bibliographystyle{IEEEtran}

% ----------------------------------------------------------------------
% Cover
% ----------------------------------------------------------------------
\begin{center}
    \begin{figure}
        \vspace{-1.0cm}
        \includegraphics[scale = 0.055]{Images/NovaFctVer.png} % Nova logo
    \end{figure}

    \mbox{}\\[2.0cm]
    \textsc{\Huge Nucleo de AeroEspacial}\\[2.5cm]
    \textsc{\LARGE ASTRO}\\[2.0cm]
    \HRule\\[0.4cm]
    {\large \bf {Electronics Report}}\\[0.2cm]
    \HRule\\[1.5cm]
\end{center}

\begin{flushleft}
    \textbf{Author:}
\end{flushleft}

\begin{center}
    \begin{minipage}{0.5\textwidth}
        \begin{flushleft}
            Lorem Ipsum (31416)\\
            Lorem Ipsum (31416)
        \end{flushleft}
    \end{minipage}%
    \begin{minipage}{0.5\textwidth}
        \begin{flushright}
            \href{mailto:lorem.ipsum@campus.fct.unl.pt}{\texttt{lorem.ipsum@campus.fct.unl.pt}}\\
            \href{mailto:lorem.ipsum@campus.fct.unl.pt}{\texttt{lorem.ipsum@campus.fct.unl.pt}}
        \end{flushright}
    \end{minipage}
\end{center}
 
\vspace{5cm}

\begin{center}
    \large \bf 2025/2026 -- \nth{1} Semester, DEEC
\end{center}

\thispagestyle{empty}

\setcounter{page}{0}

\newpage

\tableofcontents % Generates the table of contents

\newpage

\listoffigures

\newpage
\printglossary[type=\acronymtype, title={List of Acronyms}]
\newpage

\section{Electronics Working Volume}
\begin{itemize}
    \item \textbf{Dimensions:} 300\,mm $\times$ 119\,mm (inner diameter)
\end{itemize}

\begin{figure}[H]
  \includegraphics[width=\linewidth]{Images/PCB_diag.png}
  \caption{View of future PCB placement}
  \label{fig:PCB_diag}
\end{figure}

\begin{center}
\begin{tabular}{ll}
\toprule
\textbf{PCB} & \textbf{Designation} \\
\midrule
PCB 1 & SRAD \\
PCB 2 & COTS \\
PCB 3 & PSU \\
\bottomrule
\end{tabular}
\end{center}

\section{Electronic Components}

\subsection{SRAD}
\begin{center}
\begin{tabular}{lccc}
\toprule
\textbf{Device} & \textbf{Voltage (V)} & \textbf{Current (mA)} & \textbf{Power (mW)} \\
\midrule
CPU STM32F411 & 1.8 -- 3.6 & ? & ? \\
6DOF IMU ICM-45686 & 3.3 & 34 & ? \\
3DOF accelerometer  ADXL375 & 2.0--3.6 & 145~\textmu A  & ? \\
3DOF magnetometer LIS2MDL & 1.71 -- 3.6 & 200~\textmu A  & ? \\
Barometer MS5607 & 1.8--3.6 V & 1.4 & ? \\
GPS module NEO-M9N & 3.6 & ? & ? \\
LoRa SX1278 Ra-02 & 1.8 -- 3.3 & 105 & ? \\
SparkFun microSD Transflash & 3.3 & 20 & ? \\
\bottomrule
\end{tabular}
\end{center}

\subsection{COTS}
\begin{center}
\begin{tabular}{lccc}
\toprule
\textbf{Device} & \textbf{Voltage (V)} & \textbf{Current (mA)} & \textbf{Power (mW)} \\
\midrule
CATS Vega & 7--24 & 100 & 321.75 \\
\bottomrule
\end{tabular}
\end{center}

\subsection{PSU}
\begin{itemize}
    \item Converts main battery voltage to 5\,V, and 3.3\,V
    \item Can be disconnected via an arming pin (to be removed before flight)
\end{itemize}

\section*{Battery System}
Electrical power is supplied by two Lithium-Polymer (LiPo) batteries:

\begin{itemize}
    \item \textbf{Main Battery (SRAD):} 7.4\,V, 2400\,mA·h
    \item \textbf{Redundancy Battery (COTS Flight Computer):} 7.4\,V, 2400\,mA·h
\end{itemize}
Gens ace 2400mAh 2S 7.4V RX Lipo Battery Pack with JST-SYP Plug

\begin{table}[H]
\centering
\begin{tabular}{|l|l|}
\hline
\textbf{Parameter} & \textbf{Value} \\ \hline
Balancer Connector Type & JST-XHR-2P \\ \hline
Brand & Gens Ace \\ \hline
Capacity (mAh) & 2400 \\ \hline
Configuration & 2S1P \\ \hline
Connector Type & JST-SYP \\ \hline
Discharge Rate (C) & / \\ \hline
Height (±2mm) & 17 \\ \hline
Is Featured Product & No \\ \hline
Length (±5mm) & 88 \\ \hline
Max Burst Discharge Rate (C) & NO \\ \hline
Net Weight (±20g) & 94 \\ \hline
Over 300Wh & No \\ \hline
Preorder Config & No \\ \hline
Store No. & M502A \\ \hline
Voltage (V) & 7.4 \\ \hline
Width (±2mm) & 29 \\ \hline
Wire Gauge & AWG20\# \\ \hline
Discharge Wire Length (mm) & 80 \\ \hline
Quantity per Box & 42 pcs/box \\ \hline
Capacity Range (mAh) & 1000–2999 \\ \hline
\end{tabular}
\caption{Specifications of Gens Ace 7.4V 2400mAh LiPo Battery}
\label{tab:gensace2400}
\end{table}

Additionally, external power can be provided via a pad connector to keep the main battery fully charged during pre-flight checks.

\section{Programming of the SRAD Computer}

The SRAD computer was developed using \textbf{STM32CubeIDE}, chosen for its comprehensive toolset, native support for STM32 microcontrollers, and integrated debugging and configuration features. Communication between the SRAD computer and external systems is currently handled via \textbf{UART} using the RX/TX interface.

Programming and debugging are performed through the \textbf{ST-LINK V2} interface, shown in Fig.~\ref{fig:USB_coms}. This interface provides a stable and reliable connection between the development PC and the STM32 microcontroller.

\begin{figure}[H]
  \centering
  \includegraphics[width=\linewidth]{Images/USB_coms.jpeg}
  \caption{ST-LINK V2 Programmer/Debugger}
  \label{fig:USB_coms}
\end{figure}

The complete hardware setup used during development is illustrated in Fig.~\ref{fig:current_setup}. 
\begin{figure}[H]
  \centering
  \includegraphics[width=\linewidth]{Images/current_setup.jpeg}
  \caption{Current development setup}
  \label{fig:current_setup}
\end{figure}

The STM32F411 development board itself is shown in Fig.~\ref{fig:STM32411}.
\begin{figure}[H]
  \centering
  \includegraphics[width=\linewidth]{Images/STM32411.jpeg}
  \caption{STM32F411 microcontroller board}
  \label{fig:STM32411}
\end{figure}

Communication between the STM32 and an Arduino Uno has been implemented for testing purposes. The STM32 transmits IMU data over UART, and the data stream is monitored on the ground station using the Arduino IDE's serial monitor.

\textbf{TODO:}
\begin{itemize}
    \item The COTS recovery system will always issue the recovery deployment signal.
    \item The COTS system takes priority and can overrule any deployment signal generated by the SRAD computer.
    \item The firmware is built on \textbf{FreeRTOS}, providing deterministic task scheduling and modular task management.
\end{itemize}

\section{Control System / Dynamics}

\subsection{Active control}

The active control system of the \emph{ASTRO Rocket} is exclusively dedicated to the deployment of airbrakes, which are used to increase aerodynamic drag and reduce velocity in order to achieve the target apogee. Other forms of active control, such as fin actuation, are prohibited by the \acrfull{euroc} regulations \citep{euroc_rules_2025}.\\

The airbrake control is implemented using \acrshort{pid} controllers running on one of the onboard microcontrollers. This controller communicates with the servo motor system, which actively deploys the airbrakes by rotating a geared mechanism. The \acrshort{pid} controller is designed based on the relationship between the \emph{drag coefficient} and the \emph{Mach number} for each airbrake deployment level. Velocity measurements are used as the primary feedback variable to determine the appropriate control action required to reach the target velocity at each altitude. The altitude is estimated by correlating the pressure measured by the onboard barometer with the \acrfull{isa} pressure model. Further details regarding the experimental implementation and results will be presented in future revisions of this document.

\subsection{Sensing System}

The sensing and control subsystem features a more sophisticated architecture. All sensors are connected to an onboard microcontroller dedicated to data handling and signal processing. The acquired data are processed through a \acrfull{mekf} to enhance state estimation accuracy and suppress measurement noise. The \acrshort{mekf} is a well-established algorithm widely used in spacecraft attitude determination systems due to its high precision \citep{Markley2023Error-Covariance}.\\

In the context of apogee estimation, the \acrshort{mekf} contributes by providing accurate attitude quaternion estimates derived from the magnetometer, gyroscope, and accelerometer measurements. When the pitch angle, inferred from the estimated quaternion, approaches zero, the rocket is considered to have reached the apex of its parabolic trajectory, indicating the apogee point.\\

\section{Ground Station}
\begin{figure}[H]
  \includegraphics[width=\linewidth]{Images/ground_station.jpeg}
  \caption{Current Ground Station}
  \label{fig:ground_station}
\end{figure}

\textbf{TODO:}
\begin{itemize}
    \item Add simulation CSV data input
    \item Display position in 3D
    \item Temperature data
    \item Airbrake percentage
    \item Gyroscope pressure data
    \item Parachute trigger (main/rogue)
    \item Ask for team feedback
    \item Rocket arming signal
    \item Time tracking
    \item Internal pressure measurement
\end{itemize}

\section{Antennas}
Communication between the ground station and the rocket will be handled using the LoRa SX1278 Ra-02 module, paired with a Yagi-Uda antenna and a Vertical End Feed.
\\
The LoRa Sx1278 Ra-02 module provides long-range communication and high interference immunity with low power consumption

\bibliography{references}

\end{document}

